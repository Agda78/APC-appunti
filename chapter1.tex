\chapter{Richiami ed approfondimenti sui sistemi di elaborazione}
In questo capitolo sarà affrontata principalmente la parte iniziale del corso che si occupa della scrittura di programmi utilizzando il linguaggio Motorola 68k. L'interesse non sarà volto alla tipologia di architettura, anche se a volte ci sarà il bisogno di specificarla, quanto al suo utilizzo effettivo nell'ambito del corso

\section{Richiami di Motorola 68k}
Le architetture con cui sono progettate i più moderni processori possono essere di 2 tipologie:
\begin{itemize}
    \item \textbf{CISC (Complex-Istruction-Set-Computer)}: dove le istruzioni a disposizione del programmatore possono essere anche più complesse (comprendono l'utilizzo anche di più istruzioni semplici), un classico esempio è la memorizzazione di un dato in una memoria, che nel caso CISC può essere effettuato tramite un singolo comando

    \item \textbf{RISC (Reduced-Istruction-Set-Computer}: L'architettura del microprocessore permette l'utilizzo di un set più ridotto di istruzioni, semplici e lineari, tali istruzioni a differenza del paradigma CISC, possono essere più veloci, ma non ripagano in termini di complessità per l'effettuazione di determinate operazioni (come nel caso della memorizzazione di una dato in memoria)
\end{itemize}

Nel nostro caso noi utilizzeremo il Motorola 68k a 16/32 bit, dove tali bit indicano la grandezza dei registri, e di conseguenza, dei bus di collegamento tra essi. L'architetture di tale microprocessore è CISC, ma noi utilizzeremo un set ridotto di tutte le funzioni messe a disposizione dal Motorola, in modo da poter avere anche la confidenza giusta per affrontare, in futuro, anche tipologie di architetture RISC.

\subsection{Architettura generale}
Quando si interagisce con le microarchitetture si lavora con vari tipologie di registri, la cui dimensione è descritta del costruttore.
I registri possono essere divisi principalmente in registri utilizzabili dal programmatore (o registri utilizzabili) e quelli che non possono essere utilizzati dal programmatore (o non utilizzabili). Tale suddivisione vi è poichè alcuni registri all'interno della microarchitettura vengono utilizzati per effettuare delle operazioni pilotate dalla CU. Tali registri sono tutti interni alla CPU (Ricordando che la cpu è formata da CU, ALU e registri interni). I registri interni utilizzabili dal programmatore sono anche chiamati \textit{registri macchina} e possono essere di vario tipo:
\begin{itemize}
    \item \textbf{Registri Dato}: Registri che sono utilizzati per conservare un determinato dato su cui vado ad operare con varie tipologie di interazioni. Tali registri sono di una tipologia general-purpose
    \item \textbf{Registri indirizzo}: Registri che sono utilizzati per conservare gli indirizzi a cui magari si vuole accedere in memoria (tipo puntatori in C/C++)
    \item \textbf{Registri Speciali}: Registri utilizzabili dal programmatore ma con funzioni diverse, ovvero:
    \begin{itemize}
        \item \textbf{PC (Program Counter)}: memorizza la posizione del prossimo comando da eseguire del nostro programma
        \item \textbf{SR(Status Register)}: registro di stato che contiene varie tipologie di informazione, come il caso di overflow, di azzeramento del risultato, di grado di esecusione (se in administrator mode e quindi con l'accesso ad A'7)
    \end{itemize}
\end{itemize}

Tra i registri a cui invece il programmatore non ha accesso vi sono:
\begin{itemize}
    \item \textbf{MA(Memory Access)}: Registro utilizzato dal processore per scrivere l'indirizzo di memoria a cui si vuole accedere
    \item \textbf{MB(Memory Buffer)}: Registro che contiene il dato che si è letto/scritto in memoria (varia in base ai valori dei segnali di write e di read gestiti dalla CU)
\end{itemize}

\subsubsection{Memoria}
La memoria per noi funziona come un blocco a cui dati in ingresso i valori dei registri MA, ed in fase di scrittura MB, ed i segnali read e write, essa effettuerà la scrittura o la lettura sulla memoria (nel caso della lettura l'MB diventa un registro di output). \MakeUppercase{è} importante sapere come i bit sono memorizzati in memoria, per tale motivo vi sono due tipologie di identificativi:
\begin{itemize}
    \item \textbf{little-endian}: nella memorizzazione di un dato binario, riorganizzato in celle lunghe dei byte, i valori più significativi vengono memorizzati nelle celle con indirizzi più alti, mentre le meno significative in quelli con indirizzi più bassi
    \item \textbf{big-endian}: nella memorizzazione di un dato binario, riorganizzato in celle lunghe 1 Byte, esso sarà memorizzato inserendo i bit più significativi nelle celle con indirizzo più basso mentre i bit meno significativi in quelle con indirizzo più alto
\end{itemize}

\MakeUppercase{è} indifferente ai fini del funzionamento del processore quale sia la tipologia di organizzazione dell'architettura. Ma bisogna averne una conoscenza perchè è importante far capire all'unità di controllo come deve trattare i dati che sta andando a prelevare. Nel caso del Motorola 68k ci troviamo a contatto con un processore little-endian.
\\
Oltre all'organizzazione dei dati un altro problema della gestione della memoria è il suo allineamento in memoria, dato dal fatto della gestione delle sue celle di byte in maniera "indipendente".
Nel caso del motorola 68k, sono consentiti gli accessi in memoria anche a porzioni diverse di byte, ma tali porzioni hanno il vincolo di poter essere obbligatoriamente o da 2 o da 4 Byte che iniziano in posizioni di memoria pari

\subsection{Istruzioni}
In generale le istruzioni offerte dalle architetture, sono formate da tre principali parti:
\begin{itemize}
    \item \textbf{Codici operativi}: Operazioni elementari implementate in base all'architettura del processore, quindi istruzioni appartenenti al processore che specificano tutti i registri ed i passaggi da effettuare per determinate operazioni

    \item \textbf{Operandi sorgente}: Valori che possono essere memorizzati sia in dei registri interni che esterni (in base alla tipologia di operazione), su cui poi i codici operativi appartenenti all'istruzione vanno a lavorare

    \item \textbf{Operandi destinazione}: Registri o locazioni di memoria in cui si va ad inserire il dato prodotto dai codici operativi in base agli operandi sorgente ricevuti. Solitamente tale operando è indiretto poichè potrebbe diventare uno dei due operandi sorgente
\end{itemize}

In generale un istruzione è una composizione di bit, essa stessa immagazzinata in memoria, in cui una parte identifica il \textbf{codice operativo} da effettuare, mentre l'altra specifica gli operandi, che nel caso di registri interni vengono identificati con il corrispettivo indirizzo, mentre nel caso di indirizzi esterni viene specificata la locazione di memoria da cui prelevare il dato. Le tipologie di indirizzamento che posso essere utilizzate per gli operandi sono:
\begin{itemize}
    \item \textbf{Diretto}: Gli operandi presenti in memoria vengono acceduti specificando l'indirizzo di memoria in maniera plane
    \item \textbf{Indiretto}: Gli operandi in memoria vengono acceduti in base al valore puntato da un registro indirizzo
    \item \textbf{Implicito}: alcuni operandi sono dichiarati in maniera implicita all'interno dell'operando (Es. PUSH D3, pusha il valore in cima allo stack)
    \item \textbf{Immediato}: il dato da inserire in una determinata destinazione è direttamente inserito all'interno dell'istruzione (es. MOV \#7,D0)
    \item \textbf{Spiazzamento}: la locazione di memoria a cui si vuole fare riferimento viene acceduta tramite uno spiazzamento rispetto ad un indirizzo di memoria
    \item \textbf{Indice + spiazzamento}: la locazione di memoria a cui si vuole fare riferimento viene acceduta tramite un indice, che identifica una determinata zona della memoria rispetto ad un indirizzo (quindi tipo spiazzamento fisso) + un possibile ulteriore spiazzamento (per capire bene immaginarsi la memorizzazione e l'accesso a valori di una matrice)
\end{itemize}

Le istruzioni che vengono implementate per una particolare architettura vengono chiamate \textbf{ISA(Istruction Set Architecture)}, e che quindi possono essere o di tipo RISC o di tipo CISC in base alle scelte di progettazione.

Le istruzione, in generale possono essere classificate nel seguente modo:

\begin{itemize}
    \item \textbf{Trasferimento dati}: Codici che mi permettono di copiare un dato da un determinato operando e spostarlo nell'altro (MOVE)
    
    \item \textbf{Aritmetiche}: effettua delle operazioni aritmetiche sugli operandi in ingresso e le memorizza in un operando destinazione. Solitamente le funzioni appartenenti a tale classe lavorano su numeri interi

    \item \textbf{Logiche}: Operazioni che vengono effettuate sulle stringhe degli operandi con una logica bit a bit, effettuata l'operazione il risultato viene inserito all'interno di una data destinazione

    \item \textbf{Scorrimento}: operazioni effetuate sugli operandi in ingresso che restituiscono lo scorrimento verso (destra o sinistra) dell'operando e lo memorizzano in una data destinazione

    \item \textbf{Confronto}: gli operandi vengono confrontati ed in base alla tipologia di controllo che voglio effettuare vado a controllare i valori dell'SR che mi interessano

    \item \textbf{Salto}: Le istruzioni di salto permettono di cambiare il PC e quindi di eseguire (o rieseguire) delle porzioni di codice a cui puntano. Le istruzioni di salto possono essere di tipo condizionato(Bcc) o non condizionato (JMP). Nel primo caso l'istruzione di salto viene effettuata solo se è vera una data condizione, mentre nel secondo caso il salto viene effettuato senza il controllo di alcuna condizione

    \item \textbf{Input/Output}: Alcune CPU sono dotate di apposite istruzioni per trasferire i dati da e verso le periferiche apposite
\end{itemize}

\subsection{Costrutti Motorola 68k}
Una volta introdotti i concetti "teorici" e tecnologici da conoscere, si possono iniziare ad osservare i principali costrutti per la programmazione con il Motorola 68k. I costrutti principali ci permettono di capire