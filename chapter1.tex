\chapter{Richiami ed approfondimenti sui sistemi di
elaborazione}
In questo capitolo sarà affrontata principalmente la parte iniziale del corso che si occupa della scrittura di programmi utilizzando il linguaggio Motorola 68k. L'interesse non sarà volto alla tipologia di architettura, anche se a volte ci sarà il bisogno di specificarla, quanto al suo utilizzo effettivo nell'ambito del corso

\section{Richiami di calcolatori elettronici}
Il Motorola 68k è un microprocessore con architettura di tipo CISC, essa è principalmente costituita da vari registri con diverse tipologie di utilizzo. Tali registri, però, non sono una caratteristica specifica dell'architettura del Motorola 68k, pertanto è buona norma introdurre l'architettura generale di vari tipologie di microprocessori. Tali caratteristiche, quindi, non sono intrinseche del solo Motorola 68k ma sono legate alla natura stessa delle varie microarchitetture dei vari processori

\subsection{Architettura generale}
Quando si interagisce con le microarchitetture si lavora con vari tipologie di registri, la cui dimensione è descritta del costruttore.
I registri possono essere divisi principalmente in registri utilizzabili dal programmatore (o registri utilizzabili) e quelli che non possono essere utilizzati dal programmatore (o non utilizzabili). Tale suddivisione vi è poichè alcuni registri all'interno della microarchitettura vengono utilizzati per effettuare delle operazioni pilotate dalla CU. Tali registri sono tutti interni alla CPU (Ricordando che la cpu è formata da CU, ALU e registri interni). I registri interni utilizzabili dal programmatore sono anche chiamati \textbf{registri macchina (o registri general-purpose)} e possono essere di vario tipo:

\begin{itemize}
    \item \textbf{Registri Dato}: Registri che sono utilizzati per conservare un determinato dato su cui vado ad operare con varie tipologie di interazioni
    \item \textbf{Registri indirizzo}: Registri che sono utilizzati per conservare gli indirizzi a cui magari si vuole accedere in memoria (tipo puntatori in C/C++)
    \item \textbf{Registri Speciali}: Registri utilizzabili dal programmatore ma con funzioni diverse, ovvero:
    \begin{itemize}
        \item \textbf{PC (Program Counter)}: memorizza la posizione del prossimo comando da eseguire del nostro programma
        \item \textbf{IR (Istruction Register)}: Contiene una copia dell'istruzione prelevata dalla memoria 
        \item \textbf{SR(Status Register)}: registro di stato che contiene varie tipologie di informazione, come il caso di overflow, di azzeramento del risultato, di grado di esecusione (se in administrator mode e quindi con l'accesso ad A'7)
    \end{itemize}
\end{itemize}

Tra i registri a cui invece il programmatore non ha accesso vi sono:
\begin{itemize}
    \item \textbf{MA(Memory Access)}: Registro utilizzato dal processore per scrivere l'indirizzo di memoria a cui si vuole accedere
    \item \textbf{MB(Memory Buffer)}: Registro che contiene il dato che si è letto/scritto in memoria (varia in base ai valori dei segnali di write e di read gestiti dalla CU)
\end{itemize}

\subsubsection{Memoria}
La memoria per noi funziona come un blocco a cui dati indicizzati. Per cui in base all'operazione che la CU va ad effettuare, modifica i valori di: MA, MB, segnale di read e di write.
Posso memorizzare i dati i memoria in vario modo, quindi posizionandoli in vario modo, tali posizioni rispettano le seguenti due tipologie di organizzazione:
\begin{itemize}
    \item \textbf{little-endian}: nella memorizzazione di un dato binario, riorganizzato in celle lunghe dei byte, i valori più significativi vengono memorizzati nelle celle con indirizzi più alti, mentre le meno significative in quelli con indirizzi più bassi
    \item \textbf{big-endian}: nella memorizzazione di un dato binario, riorganizzato in celle lunghe 1 Byte, esso sarà memorizzato inserendo i bit più significativi nelle celle con indirizzo più basso mentre i bit meno significativi in quelle con indirizzo più alto
\end{itemize}

\MakeUppercase{è} indifferente ai fini del funzionamento del processore quale sia la tipologia di organizzazione dei dati in memoria. Ma bisogna averne una conoscenza perchè è importante far capire all'unità di controllo come deve trattare i dati che sta andando a prelevare. Nel caso del Motorola 68k ci troviamo a contatto con un processore con organizzazione big-endian.
\\
Oltre all'organizzazione dei dati un altro problema della gestione della memoria è il suo allineamento in memoria, dato dal fatto della gestione delle sue celle di byte in maniera "indipendente".
Nel caso del motorola 68k, sono consentiti gli accessi in memoria anche a porzioni diverse di byte, ma tali porzioni hanno il vincolo di poter essere obbligatoriamente o da 2 o da 4 Byte che iniziano in posizioni di memoria pari. Quindi si possono avere degli errori se si accede a posizioni di memoria dispari
\\
A volte quando si lavora con gli indirizzi di memoria si può andare anche in contro al problema dell'\textbf{Aliasing}, ovvero un problema che riguarda l'accesso a locazioni di memoria sbagliate rispetto a quelle effettivamente desiderate (nel caso del 68k questo problema è dovuto al fatto che possiede 24 fili di bus ma i registri sono a 32 bit, per cui se prelevo un indirizzo dalla memoria, perdendo gli 8 bit più significativi, se facco accesso all'indirizzo caricato, posso confonderlo con uno più piccolo)
\\
\\
La memoria, quindi memorizza varie tipologie di dati e di istruzioni. Pertanto è corretto dividere la memoria rispetto a queste due parti, per cui, la memoria sarà organizzata da due principali parti:

\begin{itemize}
    \item \textbf{Area codice (o area istruzioni)}: Area dove sono contenuti i programmi e le istruzioni da eseguire
    \item \textbf{Area dati}: Area dove sono memorizzati i dati
\end{itemize}

In generale le aree minori sono le aree codice, mentre le restanti sono aree dati

\subsection{Istruzioni}\label{par:istruzioni}
In generale le istruzioni offerte dalle architetture, sono formate da tre principali parti:
\begin{itemize}
    \item \textbf{Codici operativi}: Operazioni elementari implementate in base all'architettura del processore, quindi istruzioni appartenenti al processore che specificano tutti i registri ed i passaggi da effettuare per determinate operazioni

    \item \textbf{Operandi sorgente}: Valori che possono essere memorizzati sia in dei registri interni che esterni (in base alla tipologia di operazione), su cui poi i codici operativi appartenenti all'istruzione vanno a lavorare

    \item \textbf{Operandi destinazione}: Registri o locazioni di memoria in cui si va ad inserire il dato prodotto dai codici operativi in base agli operandi sorgente ricevuti. Solitamente tale operando è indiretto poichè potrebbe diventare uno dei due operandi sorgente
\end{itemize}

In generale un istruzione è una composizione di bit, essa stessa immagazzinata in memoria, in cui una parte identifica il \textbf{codice operativo} da effettuare, mentre l'altra specifica gli operandi, che nel caso di registri interni vengono identificati con il corrispettivo indirizzo, mentre nel caso di indirizzi esterni viene specificata la locazione di memoria da cui prelevare il dato. Le tipologie di indirizzamento che posso essere utilizzate per gli operandi sono:
\begin{itemize}
    \item \textbf{Diretto}: Gli operandi presenti in memoria vengono acceduti specificando l'indirizzo di memoria in maniera plane
    \item \textbf{Indiretto}: Gli operandi in memoria vengono acceduti in base al valore puntato da un registro indirizzo
    \item \textbf{Implicito}: alcuni operandi sono dichiarati in maniera implicita all'interno dell'operando (Es. PUSH D3, pusha il valore in cima allo stack)
    \item \textbf{Immediato}: il dato da inserire in una determinata destinazione è direttamente inserito all'interno dell'istruzione (es. MOV \#7,D0)
    \item \textbf{Spiazzamento}: la locazione di memoria a cui si vuole fare riferimento viene acceduta tramite uno spiazzamento rispetto ad un indirizzo di memoria
    \item \textbf{Indice + spiazzamento}: la locazione di memoria a cui si vuole fare riferimento viene acceduta tramite un indice, che identifica una determinata zona della memoria rispetto ad un indirizzo (quindi tipo spiazzamento fisso) + un possibile ulteriore spiazzamento (per capire bene immaginarsi la memorizzazione e l'accesso a valori di una matrice)
\end{itemize}

% Da sistemare questa parte
Le istruzioni che vengono implementate per una particolare architettura vengono chiamate \textbf{ISA(Istruction Set Architecture)}, e che quindi possono essere o di tipo RISC o di tipo CISC in base alle scelte di progettazione.

Le architetture con cui sono progettate i più moderni processori possono essere di 2 tipologie:
\begin{itemize}
    \item \textbf{CISC (Complex-Istruction-Set-Computer)}: dove le istruzioni a disposizione del programmatore possono essere anche più complesse (comprendono l'utilizzo anche di più istruzioni semplici), un classico esempio è la memorizzazione di un dato in una memoria, che nel caso CISC può essere effettuato tramite un singolo comando

    \item \textbf{RISC (Reduced-Istruction-Set-Computer}: L'architettura del microprocessore permette l'utilizzo di un set più ridotto di istruzioni, semplici e lineari, tali istruzioni a differenza del paradigma CISC, possono essere più veloci, ma non ripagano in termini di complessità per l'effettuazione di determinate operazioni (come nel caso della memorizzazione di una dato in memoria)
\end{itemize}

Nel nostro caso noi utilizzeremo il Motorola 68k a 16/32 bit, dove tali bit indicano la grandezza dei registri, e di conseguenza, dei bus di collegamento tra essi. L'architettura di tale microprocessore è CISC, ma noi utilizzeremo un set ridotto di tutte le funzioni messe a disposizione dall'M68k, in modo da poter avere anche la confidenza giusta per affrontare, in futuro, anche tipologie di architetture RISC.

Per l'esecuzione di una particolare istruzione, il microprocessore deve, prima prelevarla dalla memoria. La specifica su quale istruzione prelevare la conserva il PC (Program Counter) che conserva l'indirizzo di memoria da cui prelevare la prossima istruzione. Una volta prelevata l'istruzione dalla memoria tramite gli indirizzi MA ed MB, questa viene caricata nell'IR, che conserverà tale istruzione durante tutto il processo di decodifica ed esecuzione delle operazioni specificate

Le istruzioni, in generale possono essere classificate nel seguente modo:

\begin{itemize}
    \item \textbf{Trasferimento dati}: Codici che mi permettono di copiare un dato da un determinato operando e spostarlo nell'altro (MOVE)
    
    \item \textbf{Aritmetiche}: effettua delle operazioni aritmetiche sugli operandi in ingresso e le memorizza in un operando destinazione. Solitamente le funzioni appartenenti a tale classe lavorano su numeri interi

    \item \textbf{Logiche}: Operazioni che vengono effettuate sulle stringhe degli operandi con una logica bit a bit, effettuata l'operazione il risultato viene inserito all'interno di una data destinazione

    \item \textbf{Scorrimento}: operazioni effetuate sugli operandi in ingresso che restituiscono lo scorrimento verso (destra o sinistra) dell'operando e lo memorizzano in una data destinazione

    \item \textbf{Confronto}: gli operandi vengono confrontati ed in base alla tipologia di controllo che voglio effettuare vado a controllare i valori dell'SR che mi interessano

    \item \textbf{Salto}: Le istruzioni di salto permettono di cambiare il PC e quindi di eseguire (o rieseguire) delle porzioni di codice a cui puntano. Le istruzioni di salto possono essere di tipo condizionato(Bcc) o non condizionato (JMP). Nel primo caso l'istruzione di salto viene effettuata solo se è vera una data condizione, mentre nel secondo caso il salto viene effettuato senza il controllo di alcuna condizione

    \item \textbf{Input/Output}: Alcune CPU sono dotate di apposite istruzioni per trasferire i dati da e verso le periferiche apposite
\end{itemize}

\section{Motorola 68k}
Una volta introdotti i concetti "teorici" e tecnologici da conoscere, si possono iniziare ad osservare i principali costrutti per la programmazione con il Motorola 68k.
Conviene, quindi, non solo capire quali sono i determinati codici per le varie tipologie di istruzioni specificate in [\ref{par:istruzioni}], ma anche come costruire i principali componenti di un linguaggio di più alto livello (di cui si presupponen una minima conoscenza) tali costrutti possono essere: cicli, blocchi di decisione, ecc.

\subsection{Registri General Purpose}
I registri General-Purpose (o registri macchina) è l'insieme dei registri che sono messi a disposizione del programmatore per scrivere vari codici in accoppiata con gli specifici codici operativi.
I registri a disposizione, nel caso del motorola 68k sono i seguenti:
\begin{itemize}
    \item \textbf{Registri Dato}: Registri D0,D1,\dots,D7
    \item \textbf{Registri Indirizzo}: Registri A0,A1,\dots,A7 ed A7' (utilizzato nel caso di privilegi alti)
    \item \textbf{Status Register (SR)}: Registro che contiene vari controlli sia sui risultati delle operazioni dell'ALU che sullo stato dell'esecuzioe (se in super-user o meno)
\end{itemize}

Tali registri sono fondamentali per l'esecuzione dei comandi come nel caso del 68k. Con tali registri saranno implementati tutti gli algoritmi utili nel resto del corso

\subsection{Codici di Spostamento dati o indirizzi}
Il principale comando che nel motorola 68k permette lo spostamento dei dati è la \lstinline|MOVE|, che può essere differentemente impostata in base ai seguenti parametri:

\begin{itemize}
    \item \textbf{Lunghezza degli operandi}: solitamente specificata con delle lettere alla fine del comando
    \item \textbf{Tipologia di indirizzamento}: La \lstinline|MOVE| è una tra le poche operazioni che ammette tutte le tipologie di indirizzamento possibili, l'unico che può portare degli errori è l'indirizzamento immediato per l'operando di destinazione, che di per se non ha senso
\end{itemize}

La caratterizzazione del comando \lstinline|MOVE| è la seguente:
\begin{lstlisting}
    *Indirizzamento diretto D1 = D0 o D1<-D0
    MOVE D0,D1

    *Indirizzamento indiretto (sorgente), diretto (destinazione)
    *D0 = (A0), D0 = contenuto del registro in posizione A0
    MOVE.W (A0),D0    

    *Indirizzamento Indiretto completo A1 = A0
    MOVE.L (A0),(A1)
    *o
    MOVEA.L (A0),(A1)
    *Indirizzamento immediato + indirizzamento Diretto
    MOVE.L #14,D0

    * Indirizzamento con spiazzamento su registro di indirizzo  
    MOVE.W 4(A0), D0    * D0 = valore all'indirizzo A0 + 4  

    * Indirizzamento con spiazzamento e registro indice  
    MOVE.L 8(A0, D1.L), D2   * D2 = valore all'indirizzo A0 + 8 + D1  

    * Indirizzamento PC relativo con spiazzamento  
    MOVE.B 6(PC), D0   * D0 = byte situato 6 byte dopo il Program Counter  

    * Push di un registro nello Stack  
    MOVE.L D0, -(A7)   * Salva D0 nello stack (decremento SP)  

    * Pop dallo Stack in un registro  
    MOVE.L (A7)+, D0   * Carica D0 con il valore in cima allo stack (incremento SP)  

    * Push di un registro di indirizzo  
    MOVEA.L A0, -(A7)  * Salva A0 nello stack  

    * Pop di un registro di indirizzo  
    MOVEA.L (A7)+, A0  * Carica A0 con il valore in cima allo stack  

    * Salvataggio multiplo nello stack  
    MOVEM.L D0-D3/A0-A2, -(A7)  * Salva piu' registri nello stack  

    * Ripristino multiplo dallo stack  
    MOVEM.L (A7)+, D0-D3/A0-A2  * Ripristina piu' registri dallo stack  

    * Ritorno da subroutine (equivalente a POP del PC)  
    RTS   * Ritorna dall'ultima subroutine chiamata  
\end{lstlisting}

Nel codice precedente sono da notare le seguenti notazioni:
\begin{itemize}
    \item \textbf{.W}: tale parte del comando permette di capire che sto lavorando con operandi lunghi 16 bit (o 2 Byte) esse sono denotate Word. Nel caso in cui non sia specificato alcun markup, allora la move è da intendersi per soli 8 bit
    \item \textbf{.L}: tale parte del comando permette di capire che sto lavorando con operandi lunghi 32 bit (o 4 Byte) esse sono denotate Long Word
    \item \textbf{\#14}: Vado ad identificare un valore immediato tramite il termine \#<valore>, che sarà convertito in binario dal compilatore e poi inserito all'interno del programma, quindi integrato all'interno della zona istruzioni della mia memoria
\end{itemize}

Come si può notare il comando \lstinline|MOVE| permette di poter utilizzare tutti i tipi di indicizzazione di memoria possibili

\newpage

\subsection{Codici Aritmetici}
Per il motorola vi sono vari codici aritmetici, che però possono lavorare solo su valori interi e quindi non valori "reali" (o codificati in IEEE 754).
I codici aritmetici più importanti, ed in generale, più presenti all'interno delle varie architetture sono i seguenti:

\subsubsection{Somma}

\begin{lstlisting}
    *Operatore di somma
    ADD #3, D0  *immediato + diretto, D0 = 3+D0
    ADD.W #3,D0 *somma con specifica grandezza valore D0 = 3+D0
    ADD.W D0,D1 *indirizzamenti diretti con D1 = D0+D1

    ADDA.L #1,A0 *somma su registri di tipo indirizzo

    ADDQ.W #1,D0 *somma di un valore immediato tra 1 e 8
\end{lstlisting}

\subsubsection{Sottrazione}

\begin{lstlisting}
    *Operatore di Sottrazione
    SUB #3, D0  *immediato + diretto, D0=D0-3
    SUB.W #3,D0 *sotterazione con specifica grandezza valore D0=D0-3
    SUB.W D0,D1 *indirizzamenti diretti con D1=D1-D0

    SUBA.L #1,A0 *Sottrazione su registri di tipo indirizzo

    SUBQ.W #1,D0 *Sottrazione di un valore immediato tra 1 e 8
\end{lstlisting}

\subsubsection{Moltiplicazione}

\begin{lstlisting}
    MULU #3, D0      * Moltiplicazione senza segno immediato + diretto, D0 = D0 * 3  
    MULU.W #3,D0     * Moltiplicazione senza segno con specifica grandezza valore, D0 = D0 * 3  
    MULU.W D0,D1     * Moltiplicazione senza segno con indirizzamento diretto, D1 = D1 * D0  

    MULS.W #3,D0     * Moltiplicazione con segno immediato + diretto, D0 = D0 * 3  
    MULS.W D0,D1     * Moltiplicazione con segno tra registri, D1 = D1 * D0  
\end{lstlisting}

\subsubsection{Divisione}

\begin{lstlisting}
    DIVU #3, D0      * Divisione senza segno immediato + diretto, D0 = D0 / 3 (quoziente in D0, resto in D1)  
    DIVU.W #3,D0     * Divisione senza segno con specifica grandezza valore, D0 = D0 / 3  
    DIVU.W D0,D1     * Divisione senza segno con indirizzamento diretto, D1 = D1 / D0  

    DIVS.W #3,D0     * Divisione con segno immediato + diretto, D0 = D0 / 3  
    DIVS.W D0,D1     * Divisione con segno tra registri, D1 = D1 / D0  
\end{lstlisting}

Come si nota dalle varie implementazioni dei codici aritmetici, questi non possono utilizzare tipologie di indirizzamento indiretto. Pertanto, prima di effettuare le operazioni aritmetiche, gli operandi di input devono essere caricati nei registri interni ed il risultato sarà poi memorizzato in uno dei due registri impiegati (Come visibile nei commenti dei vari comandi).

\subsection{Codici di salto} \label{par:salto}

I codici di salto possono essere di 3 tipologie principali nel motorola 68k:
\begin{itemize}
    \item \textbf{Salti condizionati}: Quando viene incontrata l'istruzione di salto, questa effettua il salto (cambiamento del PC) in maniera immediata e senza la verifica di alcuna condizione
    
    \item \textbf{Salti incondizionati}: I salti condizionati sono effettuati in base al verificarsi di una determinata condizione. Nel caso del motorola 68k la condizione è associata ai valori associati ai singoli bit dello Statur Register (SR)
    
    \item \textbf{Salti a subrutine}: I salti a subroutine sono delle tipologie particolari di salto incondizionato, con l'unica differenza che l'indirizzo di memoria da cui si è saltati viene prima memorizzato nello stack e poi viene effettuato il salto. Tale operazione fa in modo che una volta eseguita la subroutine, il sistema possa ritornare al punto a cui si era fermato nel programma principale, senza che il programmatore debba gestire direttamente tale condizione
\end{itemize}

In generale, quando si definisce un codice di salto, bisogna prevedere anche il suo operando, che dal lato del programmatore può essere di principalmente 2 tipi:
\begin{itemize}
    \item \textbf{Label}: dopo l'istruzione si fa riferimento ad una label all'interno del programma scritto che permette di evitare di andare a lavorare con indirizzi di memoria diretti (sarà il compilatore a configurarli ad-hoc)
    
    \item \textbf{Indirizzamento indiretto}: Tramite dei registri indirizzo si indica la locazione di memoria specifica a cui si vuole saltare
\end{itemize}

\newpage
\subsubsection{Salti non condizionati}
I salti non condizionati in motorola possono essere i seguenti:
\begin{lstlisting}
    BRA label      * Salto incondizionato alla label specificata (branch always)
    JMP address    * Salto incondizionato all'indirizzo specificato
    JMP (A0)       * Salto all'indirizzo contenuto in A0
\end{lstlisting}

Solitamente nelle varie applicazioni si preferisce utilizzare il comando BRA, poichè più semplice da ricordare in riferimento ai comandi di salto condizionato

\subsubsection{Salti condizionati}
I salti condizionati hanno una forma più o meno eguale in base a quello che si vuole fare. La loro forma nel caso del 68k è del tipo: Bcc. Dove la B sta per BRANCH mentre "cc" sono le componenti che permettono di distinguere la condizione da considerare rispetto ai valori dello SR.
I principali comandi sono i seguenti:
\begin{lstlisting}
    BCS label      * Salto se Carry e' settato (C = 1)
    BCC label      * Salto se Carry e' azzerato (C = 0)
    BVS label      * Salto se Overflow e' settato (V = 1)
    BVC label      * Salto se Overflow e' azzerato (V = 0)
    BEQ label      * Salto se Zero e' settato (Z = 1)
    BNE label      * Salto se Zero e' azzerato (Z = 0)
    BMI label      * Salto se Negativo e' settato (N = 1)
    BPL label      * Salto se Positivo e' settato (N = 0)

    BLT label      * Salto se Minore di (N XOR V = 1)
    BLE label      * Salto se Minore o uguale ((N XOR V) + Z = 1)
    BGT label      * Salto se Maggiore ((N XOR V) + Z = 0)
    BGE label      * Salto se Maggiore o uguale (N XOR V = 0)

    BLS label      * Salto se Minore o uguale (C + Z = 1) (senza segno)
    BHI label      * Salto se Maggiore (C + Z = 0) (senza segno)
\end{lstlisting}

\newpage

\subsubsection{Salti a subroutine}
I salti a subroutine sono una tipologia di salto incondizionato. Tali salti fanno parte di architetture CISC principalmente, poichè alcune architetture RISC non prevedono tali funzioni. La presenza di tali funzioni permette di non dover gestire l'indirizzo di ritorno dalla subroutine, o almeno non direttamente dal programmatore. Le istruzioni che in motorola 68k sono principalmente utilizzate per la chiamata a subroutine sono le seguenti:
\begin{lstlisting}
    *Salta a una subroutine e salva il ritorno nello stack
    BSR label   

    *Salta a una subroutine con indirizzo specifico o label
    JSR address/label
    
    * Ritorna dalla subroutine (estrae l'indirizzo di ritorno dallo stack)
    RTS            
\end{lstlisting}

\subsection{Codici Logici}
I codici logici sono operazioni che possono essere effettuate su degli operandi operando bit a bit. Esempi di codici logici sono i seguenti:
\begin{lstlisting}
    AND #3, D0      * AND bit a bit con valore immediato, D0 = D0 & 3  
    AND.W D0, D1    * AND tra registri, D1 = D1 & D0  
    AND.L (A0), D0  * AND tra valore in memoria puntato da A0 e D0  

    OR #3, D0       * OR bit a bit con valore immediato, D0 = D0 | 3  
    OR.W D0, D1     * OR tra registri, D1 = D1 | D0  
    OR.L (A0), D0   * OR tra valore in memoria puntato da A0 e D0  

    EOR #3, D0      * XOR bit a bit con valore immediato, D0 = D0 XOR 3  
    EOR.W D0, D1    * XOR tra registri, D1 = D1 XOR D0  
    EOR.L (A0), D0  * XOR tra valore in memoria puntato da A0 e D0  

    NOT D0          * Complemento bit a bit (negazione), D0 = ~D0  
\end{lstlisting}

\newpage

\subsection{Codici di Scorrimento}
I codici di scorrimento permettono di effettuare delle operazioni di shift, che possono essere comode in alcune tipologie di operazioni
\begin{lstlisting}
    ASL #1, D0      * Shift aritmetico a sinistra di 1 bit (mantiene il segno)  
    ASR #1, D0      * Shift aritmetico a destra di 1 bit  

    LSL #1, D0      * Shift logico a sinistra di 1 bit (riempie con 0)  
    LSR #1, D0      * Shift logico a destra di 1 bit  

    ROL #1, D0      * Rotazione a sinistra di 1 bit (il bit piu' alto rientra da destra)  
    ROR #1, D0      * Rotazione a destra di 1 bit  

    ROXL #1, D0     * Rotazione a sinistra con Carry  
    ROXR #1, D0     * Rotazione a destra con Carry  
\end{lstlisting}

\subsection{Codici di Confronto} \label{par:confronto}
I codici di confronto sono molto importanti, poichè in accoppiata con i salti condizionati permettono di costruire tutti i costrutti fondamentali che possiamo trovare anche nei linguaggi di alto livello
\begin{lstlisting}
    CMP #5, D0      * Confronta D0 con 5 (D0 - 5, senza modificare D0, aggiorna i flag)  
    CMP.W D0, D1    * Confronta D1 con D0 (D1 - D0, aggiorna solo i flag)  
    CMP.L (A0), D0  * Confronta D0 con il valore in memoria puntato da A0  

    CMPI #10, D0    * Confronto immediato con D0 (D0 - 10, aggiorna solo i flag)  

    CMPA.L #1000, A0  * Confronta registro indirizzo A0 con 1000  
\end{lstlisting}

Oltre a semplici comparazioni, solitamente, vi sono anche dei comandi che operano sui singoli registri. Non solo per il controllo di tali registri ma anche per l'effettuazione di eventuali operazioni che possono essere comode per una tipologia di interpretazione ad alto livello del codice
\begin{lstlisting}
    TST D0          * Testa D0 (controlla se e' zero o negativo, senza modificarlo)  
    TST.W (A0)      * Testa il valore in memoria puntato da A0  

    BTST #3, D0     * Testa il bit 3 di D0 (imposta Z se il bit e' 0)  
    BTST #5, (A0)   * Testa il bit 5 della memoria puntata da A0  

    BSET #3, D0     * Imposta il bit 3 di D0 a 1  
    BSET #5, (A0)   * Imposta il bit 5 della memoria puntata da A0  

    BCLR #3, D0     * Azzera il bit 3 di D0  
    BCLR #5, (A0)   * Azzera il bit 5 della memoria puntata da A0  

    BCHG #3, D0     * Inverte il bit 3 di D0 (0 -> 1, 1 -> 0)  
    BCHG #5, (A0)   * Inverte il bit 5 della memoria puntata da A0  

    TAS D0          * Testa e imposta il bit piu' alto (7) di D0  
    TAS (A0)        * Testa e imposta il bit 7 del valore in memoria puntato da A0  
\end{lstlisting}

\subsection{Strutture sintattiche fondamentali}
Dati i codici di \textbf{Salto} [\ref{par:salto}] e quelli di \textbf{Confronto} [\ref{par:confronto}], si possono costruire quelle che sono le struttura sintattiche fondamentali

\subsubsection{if-then-else}
Per costruire il ciclo if-then-else bisogna per prima cosa comprendere quale sia la condizione, poichè bisognerà identificare:
\begin{itemize}
    \item \textbf{Registro target}: In base a quale registro/operazione devo decretare la condizione?
    \item \textbf{Condizione}: Qual'è la condizione da rispettare?
\end{itemize}

Scelti questi due parametri allora sarò capace di capire quale codice cmp utilizzare ed in che modo, e quale tipologia di salto condizionato andare ad effettuare (cerca un uguaglianza a 0, una maggiorazione, una minorazione, cosa sto cercando? quale operazione?)

Esempio di un classico If-then:
\begin{lstlisting}
                MOVE.L  D0, D1       * Carica valori nei registri (supponiamo che D0 e D1 abbiano gia' valori)
                CMP.L   D1, D0       * Confronta D0 con D1 (D0 - D1)
                BGT     END_IF   * Se D0 > D1, salta al blocco then (condizione = (D1 <= D0))
THEN:           * Codice interno all'IF

END_IF:         * Codice successivo...
\end{lstlisting}

Esempio di un If-Then-else
\begin{lstlisting}
                MOVE.L  D0, D1       * Carica valori nei registri (supponiamo che D0 e D1 abbiano gia' valori)
                CMP.L   D1, D0       * Confronta D0 con D1 (D0 - D1)
                BGT     THEN_BLOCK   * Se D0 > D1, salta al blocco THEN

                * ELSE block
                MOVE.L  #0, D2       * D2 = 0
                BRA     END_IF       * Salta oltre il blocco THEN per evitare di eseguirlo

THEN_BLOCK:     MOVE.L  #1, D2       * D2 = 1

END_IF:         * Codice successivo...
\end{lstlisting}

Come possiamo notare dal codice dell'If-then-else, abbiamo un insieme di salti sia condizionati che non condizionati che sono pilotati da una specifica istruzione di compare

\subsubsection{Ciclo FOR}
Come per l'if il ciclo for è composto principalmente da codici di \textbf{salto} e da codici di \textbf{confronto}. La struttura è molto simile a quella dell'If-Then-Else con l'eccezzione della posizione dei vari salti.
Precisamente la struttura di un ciclo for è la seguente:
\begin{lstlisting}
                MOVE.L  #0, D0         * Inizializza il contatore D0 = 0

FOR_LOOP:       CMP.L   #10, D0        * Confronta D0 con 10
                BGE     FOR_END        * Se D0 >= 10, esce dal ciclo

                * Corpo del ciclo
                NOP                    * Istruzione di esempio (da sostituire con il codice reale)

                ADDQ.L  #1, D0         * Incrementa D0 di 1
                BRA     FOR_LOOP       * Ripete il ciclo

FOR_END:        * Codice successivo dopo il ciclo

\end{lstlisting}

\subsubsection{Ciclo While}

Il ciclo while segue le regole del ciclo for solo con una condizione differente:
\begin{lstlisting}
WHILE_LOOP:     CMP.L   #0, D1        * Confronta D1 con 0
                BLE     WHILE_END     * Se D1 <= 0, esce dal ciclo

                * Corpo del ciclo
                NOP                   * Istruzione di esempio

                SUBQ.L  #1, D1        * Decrementa D1 di 1
                BRA     WHILE_LOOP    * Ripete il ciclo

WHILE_END:     * Codice successivo dopo il ciclo
\end{lstlisting}


\subsubsection{Chiamata a subroutine}

Le chiamate a subroutine possono essere viste come una sorta di chiamate a funzione. Esse, quindi, possono avere sia degli operandi di ingresso che degli operandi di uscita. La "comunicazione" degli operandi con la subroutine può avvenire in due principali modi:
\begin{itemize}
    \item \textbf{Con registri interni}: Gli operandi vengono caricati nei registri interni prima di chiamare la subroutine, che poi ci lavorerà sopra. Quindi i registri interni vengono utilizzati come una sorta di comunicazione
    \item \textbf{Con Stack}: Gli operandi sono locati sullo stack, ciò richiede quindi una gestione anche del puntatore dello stack SP
\end{itemize}

Un esempio di chiamata a subroutine con memorizzazione degli operandi nello stack è il seguente:
\begin{lstlisting}
            MOVE.L  #5, D0        * Carica il primo operando in D0
            MOVE.L  #10, D1       * Carica il secondo operando in D1

            MOVE.L  D0, -(A7)     * Push del primo operando nello stack
            MOVE.L  D1, -(A7)     * Push del secondo operando nello stack

            JSR     SUM_SUB       * Chiamata alla subroutine

            MOVE.L  (A7)+, D2     * Il chiamante preleva il risultato dallo stack

            ADDQ.L  #8, A7        * Pulizia dello stack (2 valori da 4 byte)

            * D2 ora contiene il risultato della somma

            * Codice successivo...

SUM_SUB:    MOVE.L  (A7)+, D0     * Pop del primo operando dallo stack
            MOVE.L  (A7)+, D1     * Pop del secondo operando dallo stack

            ADD.L   D1, D0        * Somma D0 + D1, risultato in D0

            MOVE.L  D0, -(A7)     * Push del risultato nello stack

            RTS                   * Ritorna al chiamante
\end{lstlisting}
\newpage
Esempio di chiamata a subroutine con operandi nello stack e risultato memorizzato nello stack
\begin{lstlisting}
            MOVE.L  #5, D0        * Carica il primo operando in D0
            MOVE.L  #10, D1       * Carica il secondo operando in D1

            MOVE.L  D0, -(A7)     * Push del primo operando nello stack
            MOVE.L  D1, -(A7)     * Push del secondo operando nello stack

            JSR     SUM_SUB       * Chiamata alla subroutine

            MOVE.L  (A7)+, D2     * Il chiamante preleva il risultato dallo stack

            ADDQ.L  #8, A7        * Pulizia dello stack (2 valori da 4 byte)

            * D2 ora contiene il risultato della somma

            * Codice successivo...

SUM_SUB:    MOVE.L  (A7)+, D0     * Pop del primo operando dallo stack
            MOVE.L  (A7)+, D1     * Pop del secondo operando dallo stack

            ADD.L   D1, D0        * Somma D0 + D1, risultato in D0

            MOVE.L  D0, -(A7)     * Push del risultato nello stack

            RTS                   * Ritorna al chiamante
\end{lstlisting}

Caso di utilizzo dei registri interni, sia per passaggio operandi di ingresso che di uscita:

\begin{lstlisting}
            MOVE.L  #5, D0       * Primo operando in D0
            MOVE.L  #10, D1      * Secondo operando in D1

            JSR     SUM_REGS     * Chiamata alla subroutine

            * Dopo il ritorno, il risultato e' in D0

            * Codice successivo...

SUM_REGS:   ADD.L   D1, D0       * Somma D0 + D1, risultato in D0

            RTS                  * Ritorna al chiamante

\end{lstlisting}

\subsection{Valutazione degli accessi in memoria}
Quando utilizzo i comandi precedentemente presentati avrò una quantità diversa di accessi in memoria in base alla composizione che ho dato al mio codice. Gli accessi in memoria dipendono fortemente dalla tipologia di architettura che ho adottato. In generale gli accessi in memoria possono avvenire per 2 tipologie di operazioni: Accesso in memoria Per le Istruzioni (PI) o accesso in memoria Per le Operazioni (PO).
Veodiamo degli esempi per capire meglio di cosa si sta parlando:
\begin{table}[h]
    \centering
    \begin{tabular}{|c|c|c|}
        \hline
        \textbf{Istruzione} & \textbf{PI} & \textbf{PO} \\ 
        \hline
        \lstinline|MOVE.L D0,D1| & 1 & 0 \\  
        \lstinline|MOVE.W D0,D1| & 1 & 0 \\  
        \lstinline|MOVE.L #7,D1| & 3 & 0 \\
        \lstinline|MOVE.W (A0),(A1)| & 1 & 2 \\
        \lstinline|MOVE.W (A0),VAR| & 3 & 2 \\ 
        \hline
    \end{tabular}
    \caption{Conteggio accessi per Architettura a 16 bit e VAR a 32 bit}
    \label{tab:esempio}
\end{table}

Per contare gli accessi in memoria bisogna effettuare delle osservazioni in base alla parte che si sta analizzando.
Per l'accesso \textbf{Per le Istruzioni (PI)} si ragiona sui seguenti accessi:
\begin{itemize}
    \item \textbf{Prelievo dell'istruzione}: Un operazione che non mancherà mai sarà sempre il prelievo dell'istruzione, che impone che il mio PI non potrà mai essere nullo
    \item \textbf{Operandi immediati}: se nel mio comando ho degli operandi immediati, allora dovrò accedere anche altre volte alla memoria per il prelievo di tale operando. I miei accessi, per questo caso, sono dettati dalla lunghezza dell'operando rispetto ai miei bus disponibili. Per esempio, se sono in un architettura a 16 bit devo prelevare un immegiato considerato una WORD, allora il numero di accessi aggiuntivi per prelevare l'immediato è uguale a 1. Mentre se stessi lavorango con le Long World (32 bit), allora il numero di accessi, a parità di architettura, sarà 2
    \item \textbf{Variabili}: se sto utilizzando delle variabili, che indicano delle locazioni di memoria dirette, allora dovrò fare un numero di accessi alla memoria che mi permette di prelevare gli indirizzi (tali indirizzi sono lunghi tutti 32 bit). Pertanto con un architettura a 16 bit dovrò considerare sempre 2 accessi per ogni variabile per prelevare tali indirizzi
\end{itemize}

Per l'accesso \textbf{Per gli Operandi (PO)}, i parametri risultano più o meno gli stessi, ad esclusione del prelievo istruzione e della variabile. Le operazioni che si contano per tale processo sono:
\begin{itemize}
    \item \textbf{Indirizzamento Indiretto}: quando vado ad effettuare dei riferimenti a dei registri della memoria con indirizzamento indiretto allora devo prevedere un numero di accessi per il prelievo dell'operando. Quindi bisogna conteggiare il numero di accessi per il prelievo dell'operando in base alla sua lunghezza. Nel caso di architettura a 16 bit si avranno: 1 accesso per prelevare delle word (16 bit) e 2 accessi pre prelevare le Long Word (32 bit)
    
    \item \textbf{Variabili}: Quando si utilizzano le variabili, oltre a prelevare gli indirizzi dalla "zona istruzioni" bisogna prelevare gli operando. La conta del numero di accessi per il prelievo degli operandi è uguale al caso di indirizzamento indiretto
\end{itemize}

Nel caso degli accessi PO, si è prevista la considerazione per singoli operandi. Quindi in base al numero di operandi presenti, la loro lunghezza prefissata, il loro modo di indirizzamento, si riesce a comprendere (cumulando le specifiche), il numero di accessi che bisogna effettuare in memoria ed il motivo di tali accessi

