\chapter{Gestione dei dispositivi di IO}
I dospositivi di input/output (o I/O), sono tutti quei dispositivi che si connettono alla classica architettura composta solo da processore e memoria centrale. Tra tali dispositivi rientrano: Memorie di massa (HDD e SSD), mouse, tastiera, sensori ecc.
Data l'eterogeneità di tali periferiche è richiesto che queste ultime siano gestite in un certo modo, o almeno, che la loro gestione principale sia di un certo tipo. Ciò, quindi, pone le basi su come dovremmo interfacciarci all'utilizzo di tali dispositivi

\section{Architettura generale di un dispositivo di I/O}
In generale un dispositivo di I/O può essere visto come l'insieme di tre parti fondamentali:
\begin{itemize}
    \item \textbf{Registri Dato, Stato e Controllo}: Tali registri sono quelli che interagiscono in maniera diretta con la CPU, e vengono utilizzati da quest'ultima per controllare e gestire le informazioni di quel dato dispositivo. Tali registri sono presenti per tutte le tipologie di periferiche
    
    \item \textbf{Sistema di adattamento}: Il sistema di adattamento adatta i segnali provenienti dal mondo esterno per essere letti o scritti nei registri di Dato, Stato e Controllo
    
    \item \textbf{Mondo esterno}: Per mondo esterno si intende tutta la parte che interagisce con il dispositivo in maniera fisica
\end{itemize}

Un esempio di dispositivo esterno è la memoria HDD.
La memoria HDD ha difatti i tre registri di Dato, Stato e Controllo, quando si vuole scrivere su tale memoria, la CPU va a modificare i registri in modo da garantire tale operazione. Mentre la CPU modifica tali dati, il sistema di adattamento converte i dati presenti in quei tre registri in movimenti della testina + scrittura, rispettando sempre i controlli dati dalla CPU. La scrittura/lettura dei dati tramite la testina e la testina stessa rappresentano, invece, il mondo esterno.
Un altro esempio di periferica è la classica porta UART, che trasmette i sui dati in serie, ma il suo controllo avviene in parallelo. Pertanto al suo interno avrà sia un timer per scandire il clock in base alla tipologia di comunicazione, e poi avrà un buffer parallelo serie, che converte l'informazione da trasmettere in tanti bit seriali. Oltre alla parte parallelo-serie sarà anche dotato di una parte serie-parallelo, nel caso della ricezione.


\subsection{Modalità di comunicazione}
Le tipologie di collegamento che si possono avere tra un processore e le sue periferiche sono le seguenti:
\begin{itemize}
    \item \textbf{Collegamento passivo}: la periferica e la CPU non condividono alcun tipo di comunicazione. Quindi la CPU presuppone che la periferica sia sempre pronta ed è quindi solo lei a decidere quando e come utilizzare i dati, anche se questi magari non sono pronti o ben processati
    \item \textbf{Collegamento Sincrono}: La periferica e la CPU comunicano tra loro, la comunicazione è sincronizzata da un clock comune
    \item \textbf{Collegamento con Handshacking}: L'handshackin è una modalità di sincronizzazione asincrona, poichè si sfruttano dei segnali di comunicazione tra la CPU e la periferica che permettono di capire quando il dato è "pronto" o meno. Una classica implementazione è quella del segnale di req che viene alzato dal processore per far capire che vuole leggere e dall'ack emanato dalla periferica che fa comprendere che il dato è pronto o che è stata presa in carico l'operazione
    \item \textbf{Collegamento semisincrono}: Si condividono le stesse modalità di una comunicazione con handshacking, con la differenza che la sincronizzazione delle due parti avviene mediante uno stesso clock
\end{itemize}

\subsection{Interfacciamento CPU e periferica}
Per utilizzare le periferiche la CPU deve poter accedere ai registri di Dato, Stato e Controllo di tali periferiche. Le tipologie di interfacciamento che ci possono essere tra CPU e Periferica sono:
\begin{itemize}
    \item \textbf{Memory Mapped I/O}: La CPU fa riferimento ai registri di Dato, Stato e Controllo di una periferica come se fossero dei registri in memoria
    \item \textbf{I/O Mapped}: La CPU ha specifici comandi per interagire con le periferiche di I/O
\end{itemize}

Nel nostro caso il Motorola 68k è una tipologia di architettura memory mapped, e quindi la trattazione dei registri avviene mediante i classici comandi di spostamento già utilizzati

\subsubsection{Memory Mapped I/O}
Nel caso di interfacciamento con una struttura Memory Mapped, l'accesso ai registri di una determinata periferica avvengono tramite i bus di collegamento classici, che collegano anche la memoria ecc. Ciò quindi mi limita nell'utilizzo degli indirizzi, poichè, quando faccio riferimento ad un registro di una periferica, tale indirizzo non deve appartenere al set di indirizzi della memoria centrale

\subsubsection{I/O Mapped}
Nel caso di interfacciamento con una struttura I/O Mapped, l'accesso ai registri di una determinata periferica avviene mediante degli specifici comandi. Questo perchè le periferiche sono collegate a bus dedicati o hanno una gestione dedicata, che quindi differisce dalle comunicazioni che avvengono in generale all'interno dell'architettura al costo di avere meno modi di indirizzamento, dato che non si userà più la MOVE che è un codice operativo ortogonale

\subsubsection{Logiche di selezione}
Quando devo selezionare la mia periferica a cui faccio riferimento, utilizzo una serie di indirizzi. Tali indirizzi possono essere utili al fine di realizzare i seguenti tipi di logica:
\begin{itemize}
    \item \textbf{Logica tristate}: Logica che quando una periferica non vede il suo indirizzo sui bus adeguati smette di interagire con il sistema, quindi ignora la variazione dei dati sul bus. Tale logica, quindi utilizza l'indirizzo interno della nostra periferica
    \item \textbf{Logica Plug-and-play}: L'indirizzo della periferica viene scelto in base ad una serie di indirizzi disponibili
\end{itemize}

\subsection{BUS}
I bus sono i collegamenti che interconnettono le varie componenti di un calcolatore, ovvero, CPU, memoria e periferiche di I/O.
In generale non vi è una tipologia unica di bus, ve ne sono varie in base alla tipologia di utilizzi e alla tipologia di tecnologie utilizzate.
I bus, si contraddistinguono principalmente per la divisione che attuano sui loro collegamenti, ma in generale, le informazioni che vengono trasportate sono solitamente le stesse.
Le informazioni, quindi, sono dipartite tra i vari collegamenti presenti in un BUS. I collegamenti generici che si possono identificare in un bus sono:
\begin{itemize}
    \item \textbf{Alimentazione}: Collegamenti che principalmente comprendono la VCC (o più VCC), che sarebbero le tensioni di alimentazione delle componenti; ed il cavo di terra (o GND)
    \item \textbf{Dati}: Collegamenti che trasportano i dati che vengono interscambiati tra i vari dispositivi
    \item \textbf{Indirizzo}: Collegamenti che trasportano gli indirizzi che permettono la selezione dei dispositivi interessati o dei registri a cui si vuole accedere
    \item \textbf{Controllo}: Collegamenti che trasportano le informazioni inerenti alla tipologia di operazione che si vuole effettuare
    \item \textbf{Stato}: Collegamenti che permettono il controllo di flusso e la segnalazione di eventuali conflitti o errori
\end{itemize}

Data una tipologia di bus, può capitare che la periferica che vado ad utilizzare non è ad-hoc per quella determinata tipologia di bus. Pertanto, quello che posso fare, è considerare l'utilizzo di un \textbf{adapter}, che mi permette di adattare il bus classico con la tipologia di attacco specifica per la mia periferica. Oltretutto in alcuni casi, quando il dispositivo non permette la configurazione degli indirizzi, per evitare conflitti, l'adapter gestisce anche la gestione di tale indirizzo rispetto al sistema

\subsection{Driver}
I driver sono dei programmi che permettono di capire come il processore vada ad utilizzare una determinata periferica.
Le tipologie di approccio che si possono avere nella scrittura dei driver sono varie, la più primitiva è il polling.
Il \textbf{Polling} è un modo con cui il processore va ad interagire con la periferica. In generale si va a dare un primo segnale di controllo alla periferica e si aspetta un'altro specifico dato di controllo si alzi per poter utilizzare o accedere al dato. Tali sistema è altamente inefficiente, poichè mentre la CPU aspetta la riposta della periferica passano dei periodi di clock dove la CPU rimane ferma. Il tempo che quindi la CPU rimane senza eseguire delle operazioni utili è detto \textbf{Busy-waiting}. Un possibile codice di implementazione del polling è [\ref{m68:polling}]

\begin{lstlisting}[caption={Codice polling}, label=m68:polling]
L1      MOVE.B  #00000000,Stato     *Imposto lo stato iniziale
        MOVE.B  #00000001,Controllo *Abilito la "richiesta"
        MOVE.B  Stato,D0              *Prelevo lo stato
        AND.B   #01000000,D0        *Il dato e' pronto?
        BNE     L1                    *Se no ripeti
        
        ORG     $2000
Dato    DS.B    1
Stato   DS.B    1
Ctr     DS.B    1

        ORG     $8000
VAR     DC.B    44
\end{lstlisting}

L'alternativa che si è trovata al polling è l'utilizzo delle interruzioni <Continuare quando lo spiegherà>
