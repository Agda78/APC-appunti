\chapter*{Introduzione}
\addcontentsline{toc}{chapter}{Introduzione}
Tali appunti sono stati scritti man mano che si seguiva il corso tenuto dal professor Mazzocca di Architettura e progettazione dei calcolatori. Sono una valida base per gli argomenti che bisogna conoscere per l'esame. L'unico dettaglio che mi sento di sottolineare è che ci sono degli argomenti che richiedono di essere approfonditi (mediante altri appunti, materiale del corso o internet). Precisamente gli argomenti che maggiormente richiedono un approfondimento sono:
\begin{itemize}
    \item Architettura del MIPS: Approfondimento sul funzionamento delle componenti interne di ogni fase della pipeline
    \item Strutturazione delle memorie: Approfondimento sulle memorie set-associative e loro composizione
    \item Speed-Up: La formula presentata a lezione è quella che è stata presentata a lezione benchè differisca da quella originale presentata in altri testi ed in rete (potrebbe creare confusione)
    \item Macchine Virtuali: Consiglio di rivedere varie cose provenienti dall'esame di sistemi operativi
\end{itemize}

