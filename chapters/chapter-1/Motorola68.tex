\section{Motorola 68k}
Una volta introdotti i concetti teorici e tecnologici propedeutici, si possono iniziare ad osservare i principali costrutti per la programmazione con il Motorola 68000.
Conviene, quindi, non solo capire quali sono i  codici per le varie tipologie di istruzioni specificate nel paragrafo [\ref{par:istruzioni}], ma anche come costruire i principali componenti di un linguaggio di più alto livello (di cui si presuppone una minima conoscenza). I costrutti in questione possono essere: cicli, blocchi di decisione, ecc.

\subsection{Registri General Purpose}
Con \textit{registri General-Purpose} (o registri macchina) intendiamo l'insieme dei registri messi a disposizione del programmatore per scrivere codice sfruttando i codici operativi dell'ISA.
I registri General-Purpose a disposizione sono un insieme di registri a 32 bit:
\begin{itemize}
    \item \textbf{Registri Dato}: Registri D0,D1,\dots,D7;
    \item \textbf{Registri Indirizzo}: Registri A0,A1,\dots,A7 ed A7' (questi ultimi puntano alla cima dello STACK: A7' è accessibile solo in modalità \textit{supervisore}. In generale, ogni modalità di esecuzione dispone di un'area stack separata);
    \item \textbf{Status Register (SR)}: Registro a 16 bit che contiene informazioni sia sui risultati delle operazioni dell'ALU che sullo stato dell'esecuzione (se in super-user o meno). Questo regsitro è sempre accessibile in lettura, mentre è accessibile in scrittura solo in modalità supervisore.
\end{itemize}

\begin{table}[h!]
\centering
\renewcommand{\arraystretch}{1.2}
\begin{tabular}{|c|c|c|}
\hline
\textbf{Bit} & \textbf{Sigla} & \textbf{Descrizione} \\
\hline
15 & T & Trace Enable \\
\hline
14 & S & Supervisor State (1=Supervisor) \\
\hline
13-8 & - & Non usati (riservati) \\
\hline
7 & X & Extend (Estensione del Carry) \\
\hline
6 & N & Negative \\
\hline
5 & Z & Zero \\
\hline
4 & V & Overflow \\
\hline
3 & C & Carry \\
\hline
2-0 & - & Non usati (riservati) \\
\hline
\end{tabular}
\caption{Bit dello Status Register (SR) del Motorola 68k}
\label{table:SR68k}
\end{table}


\subsection{Codici di Spostamento dati o indirizzi}
Il principale codice operativo che nel motorola 68k permette lo spostamento dei dati è la \lstinline|MOVE|, che può essere differentemente utilizzata in base ai seguenti parametri:

\begin{itemize}
    \item \textbf{Lunghezza degli operandi}: solitamente specificata con la \textit{dot notation} alla fine del comando;
    \item \textbf{Tipologia di indirizzamento}: La \lstinline|MOVE| è una tra le poche operazioni che ammette tutte le tipologie di indirizzamento possibili (proprietà detta di \textbf{ortogonalità}); L'indirizzamento immediato per l'operando di destinazione è l'unico che può causa errore, ma è un'operazione che comunque non ha alcun senso.
\end{itemize}

La caratterizzazione del comando \lstinline|MOVE| è la seguente:
\begin{lstlisting}
    *Indirizzamento diretto D1 = D0 o D1<-D0
    MOVE D0,D1

    *Indirizzamento indiretto (sorgente), diretto (destinazione)
    *D0 = (A0), D0 = contenuto del registro in posizione A0
    MOVE.W (A0),D0

    *Indirizzamento Indiretto completo
    MOVE.L (A0),(A1) *(istruzione non valida)
    *o
    MOVEA.L (A0),A1 *(istruzione valida)
    *Indirizzamento immediato + indirizzamento Diretto
    MOVE.L #14,D0

    * Indirizzamento con spiazzamento su registro di indirizzo
    MOVE.W 4(A0), D0    * D0 = valore all'indirizzo A0 + 4

    * Indirizzamento con spiazzamento e registro indice
    MOVE.L 8(A0, D1.L), D2   * D2 = valore all'indirizzo A0 + 8 + D1

    * Indirizzamento PC relativo con spiazzamento
    MOVE.B 6(PC), D0   * D0 = byte situato 6 byte dopo il Program Counter

    * Push di un registro nello Stack
    MOVE.L D0, -(A7)   * Salva D0 nello stack (decremento SP)

    * Pop dallo Stack in un registro
    MOVE.L (A7)+, D0   * Carica D0 con il valore in cima allo stack (incremento SP)

    * Push di un registro di indirizzo
    MOVEA.L A0, -(A7)  * Salva A0 nello stack

    * Pop di un registro di indirizzo
    MOVEA.L (A7)+, A0  * Carica A0 con il valore in cima allo stack

    * Salvataggio multiplo nello stack
    MOVEM.L D0-D3/A0-A2, -(A7)  * Salva piu' registri nello stack

    * Ripristino multiplo dallo stack
    MOVEM.L (A7)+, D0-D3/A0-A2  * Ripristina piu' registri dallo stack

    * Ritorno da subroutine (equivalente a POP del PC)
    RTS   * Ritorna dall'ultima subroutine chiamata
\end{lstlisting}

\textbf{Nota sull'indirizzamento indiretto completo}: Non è possibile utilizzare un'operazione del tipo \lstinline|MOVE (A0), (A1)| perchè il bus dati dell'architettura del MC68k non supporta simultaneamente operazioni di lettura e scrittura, dunque servono due cicli di bus distinti. Occorre utilizzare dei registri dato interni di appoggio.

\textbf{Nota su} \lstinline|MOVE.L D0, -(A7)|: il pre-decremento nell'operazione di push nello stack è dovuto al fatto che, per costruzione, lo stack "cresce" verso il basso, quindi deve decrescere di 4 byte per ospitare quanto vi sto inserendo (quanto detto è scollegato dal fatto che il 68k sia big endian).

\textbf{Nota su RTS}: come concetto, RTS equivale a MOVE.L (A7)+, PC perché il valore puntato da A7 è quello * in cima allo stack, in questo caso estraiamo quindi + (lo stack "sale"), in caso contrario avremmo avuto - (quando inseriamo nello stack).

Nel codice precedente sono da notare le seguenti notazioni:
\begin{itemize}
    \item \textbf{.W}: tale parte del comando permette di capire che sto lavorando con operandi lunghi 16 bit (o 2 Byte) esse sono denotate Word. Nel caso in cui non sia specificato alcun markup, allora la move è da intendersi per soli 8 bit
    \item \textbf{.L}: tale parte del comando permette di capire che sto lavorando con operandi lunghi 32 bit (o 4 Byte) esse sono denotate Long Word
    \item \textbf{\#14}: Vado ad identificare un valore immediato tramite il termine \#<valore>, che sarà convertito in binario dal compilatore e poi inserito all'interno del programma, quindi integrato all'interno della zona istruzioni della mia memoria
\end{itemize}

\newpage

\subsection{Codici Aritmetici}
Per il motorola vi sono vari codici aritmetici, che però possono lavorare solo su valori interi e quindi non valori "reali" (o codificati in IEEE 754).
I codici aritmetici più importanti, ed in generale, più presenti all'interno delle varie architetture sono i seguenti:

\subsubsection{Somma}

\begin{lstlisting}
    *Operatore di somma
    ADD #3, D0  *immediato + diretto, D0 = 3+D0
    ADD.W #3,D0 *somma con specifica grandezza valore D0 = 3+D0
    ADD.W D0,D1 *indirizzamenti diretti con D1 = D0+D1

    ADDA.L #1,A0 *somma su registri di tipo indirizzo, somma 1 direttamente al contenuto di A0, non alla locazione di memoria puntata da A0 (NON ci sonon le parentesi)

    ADDQ.W #1,D0 *somma di un valore immediato tra 1 e 8
\end{lstlisting}

\subsubsection{Sottrazione}

\begin{lstlisting}
    *Operatore di Sottrazione
    SUB #3, D0  *immediato + diretto, D0=D0-3
    SUB.W #3,D0 *sotterazione con specifica grandezza valore D0=D0-3
    SUB.W D0,D1 *indirizzamenti diretti con D1=D1-D0

    SUBA.L #1,A0 *Sottrazione su registri di tipo indirizzo

    SUBQ.W #1,D0 *Sottrazione di un valore immediato tra 1 e 8
\end{lstlisting}

\subsubsection{Moltiplicazione}

\begin{lstlisting}
    MULU #3, D0      * Moltiplicazione senza segno immediato + diretto, D0 = D0 * 3
    MULU.W #3,D0     * Moltiplicazione senza segno con specifica grandezza valore, D0 = D0 * 3
    MULU.W D0,D1     * Moltiplicazione senza segno con indirizzamento diretto, D1 = D1 * D0

    MULS.W #3,D0     * Moltiplicazione con segno immediato + diretto, D0 = D0 * 3
    MULS.W D0,D1     * Moltiplicazione con segno tra registri, D1 = D1 * D0
\end{lstlisting}

\subsubsection{Divisione}

\begin{lstlisting}
    DIVU #3, D0      * Divisione senza segno immediato + diretto, D0 = D0 / 3 (quoziente in D0, resto in D1)
    DIVU.W #3,D0     * Divisione senza segno con specifica grandezza valore, D0 = D0 / 3
    DIVU.W D0,D1     * Divisione senza segno con indirizzamento diretto, D1 = D1 / D0

    DIVS.W #3,D0     * Divisione con segno immediato + diretto, D0 = D0 / 3
    DIVS.W D0,D1     * Divisione con segno tra registri, D1 = D1 / D0
\end{lstlisting}

Come si nota dalle varie implementazioni dei codici aritmetici, questi non possono utilizzare tipologie di indirizzamento indiretto. Pertanto, prima di effettuare le operazioni aritmetiche, gli operandi di input devono essere caricati nei registri interni ed il risultato sarà poi memorizzato in uno dei due registri impiegati (Come visibile nei commenti dei vari comandi).
Inoltre, sia MULU e MULS che DIVU e DIVS lavorano implicitamente su 16 bit nel 68k, e non esistono cose del tipo MULU.B oppure MULU.L (idem per gli altri 3 codici operativi).

\subsection{Codici di salto} \label{par:salto}

I codici di salto possono essere di 3 tipologie principali nel motorola 68k:
\begin{itemize}
    \item \textbf{Salti incondizionati}: Quando viene incontrata l'istruzione di salto, questa effettua il salto (cambiamento del PC) in maniera immediata e senza la verifica di alcuna condizione;

    \item \textbf{Salti condizionati}: I salti condizionati sono effettuati in base al verificarsi di una determinata condizione. Nel caso del motorola 68k la condizione è associata ai valori dei singoli bit dello Statur Register (SR);

    \item \textbf{Salti a subrutine}: I salti a subroutine sono delle tipologie particolari di salto incondizionato, con l'unica differenza che l'indirizzo di memoria da cui si è saltati viene prima memorizzato nello stack e poi viene effettuato il salto. Tale operazione fa in modo che una volta eseguita la subroutine, il sistema possa ritornare al punto a cui si era fermato nel programma principale, sollevando il programmatore dall'onere di gestire manualmente il salvataggio dell'indirizzo di ritorno.
\end{itemize}

In generale, quando si definisce un codice di salto, bisogna prevedere anche il suo operando, che dal lato del programmatore può essere di principalmente 2 tipi:
\begin{itemize}
    \item \textbf{Label}: dopo l'istruzione si fa riferimento ad una label all'interno del programma scritto che permette di evitare di andare a lavorare con indirizzi di memoria diretti (sarà il compilatore a configurarli ad-hoc);

    \item \textbf{Indirizzamento indiretto}: Tramite dei registri indirizzo si indica la locazione di memoria specifica a cui si vuole saltare;
\end{itemize}

\newpage
\subsubsection{Salti non condizionati}
I salti non condizionati in motorola possono essere i seguenti:
\begin{lstlisting}
    BRA label      * Salto incondizionato alla label specificata (branch always)
    JMP address    * Salto incondizionato all'indirizzo specificato
    JMP (A0)       * Salto all'indirizzo contenuto in A0
\end{lstlisting}

Solitamente nelle varie applicazioni si preferisce utilizzare il comando BRA, poichè più semplice da ricordare in riferimento ai comandi di salto condizionato.

\subsubsection{Salti condizionati}
I salti condizionati hanno una forma più o meno uguale in base a quello che si vuole fare. La loro forma nel caso del 68k è del tipo: Bcc. Dove la B sta per BRANCH mentre "cc" sono le componenti che permettono di distinguere la condizione da considerare rispetto ai valori dello SR. In particolare, i salti condizionati operano in base ai valori dei 5 bit del \textbf{CCR (Condition Code Register)}, che sono i 5 bit meno significativi dello SR riportati in tabella \ref{table:SR68k}.

I principali comandi sono i seguenti:
\begin{lstlisting}
    BCS label      * Salto se Carry e' settato (C = 1)
    BCC label      * Salto se Carry e' azzerato (C = 0)
    BVS label      * Salto se Overflow e' settato (V = 1)
    BVC label      * Salto se Overflow e' azzerato (V = 0)
    BEQ label      * Salto se Zero e' settato (Z = 1)
    BNE label      * Salto se Zero e' azzerato (Z = 0)
    BMI label      * Salto se Negativo e' settato (N = 1)
    BPL label      * Salto se Positivo e' settato (N = 0)

    BLT label      * Salto se Minore di (N XOR V = 1)
    BLE label      * Salto se Minore o uguale ((N XOR V) + Z = 1)
    BGT label      * Salto se Maggiore ((N XOR V) + Z = 0)
    BGE label      * Salto se Maggiore o uguale (N XOR V = 0)

    BLS label      * Salto se Minore o uguale (C + Z = 1) (senza segno)
    BHI label      * Salto se Maggiore (C + Z = 0) (senza segno)
\end{lstlisting}

\newpage

\subsubsection{Salti a subroutine}
I salti a subroutine sono una tipologia di salto incondizionato. Tali salti fanno parte di architetture CISC principalmente, poichè alcune architetture RISC non prevedono tali funzioni. La presenza di tali funzioni permette di non dover gestire l'indirizzo di ritorno dalla subroutine, o almeno non direttamente dal programmatore. Le istruzioni che in motorola 68k sono principalmente utilizzate per la chiamata a subroutine sono le seguenti:
\begin{lstlisting}
    * Salta a una subroutine e salva il ritorno nello stack 
    * (push nello stack)
    BSR label
    *Salta a una subroutine con indirizzo specifico o label
    JSR address/label
    * Ritorna dalla subroutine (estrae l'indirizzo di 
    * ritorno dallo stack, fa il pop dallo stack)
    RTS
\end{lstlisting}

\subsection{Codici Logici}
I codici logici sono operazioni che possono essere effettuate su degli operandi lavorando bit a bit. Da notare che per i codici logici si può avere l'indirizzamento indiretto solo per la sorgente e non per il destinatario. Esempi di codici logici sono i seguenti:
\begin{lstlisting}
    AND #3, D0      * AND bit a bit con valore immediato, D0=D0&3
    AND.W D0, D1    * AND tra registri, D1=D1&D0
    AND.L (A0), D0  * AND tra valore puntato da A0 e D0

    OR #3, D0       * OR bit a bit con valore immediato, D0=D0|3
    OR.W D0, D1     * OR tra registri, D1=D1|D0
    OR.L (A0), D0   * OR tra valore in memoria puntato da A0 e D0

    EOR #3, D0      * XOR bit a bit con valore immediato, D0=D0XOR3
    EOR.W D0, D1    * XOR tra registri, D1 = D1 XOR D0
    EOR.L (A0), D0  * XOR tra valore puntato da A0 e D0

    NOT D0          * Complemento bit a bit (negazione), D0 = ~D0
\end{lstlisting}

\newpage

\subsection{Codici di Scorrimento}
I codici di scorrimento permettono di effettuare delle operazioni di shift, che possono essere comode in alcune tipologie di operazioni
\begin{lstlisting}
    ASL #1, D0      * Shift aritmetico a sinistra di 1 bit 
                    *(mantiene il segno)
    ASR #1, D0      * Shift aritmetico a destra di 1 bit

    LSL #1, D0      * Shift logico a sinistra di 1 bit (0 fill)
    LSR #1, D0      * Shift logico a destra di 1 bit

    ROL #1, D0      * Rotazione a sinistra di 1 bit 
                    * (il bit piu' alto rientra da destra)
    ROR #1, D0      * Rotazione a destra di 1 bit

    ROXL #1, D0     * Rotazione a sinistra con Carry
    ROXR #1, D0     * Rotazione a destra con Carry
\end{lstlisting}

Osserviamo che lo shift aritmetico a sinistra ha l'effetto di moltiplicare per 2 (se convertiamo da decimale a binario il valore del registro prima e dopo lo shift), tranne se perdi bit significativi o alteri il segno, e dopo lo shift aritmetico possono essere modificati i bit del CCR.

\subsection{Codici di Confronto} \label{par:confronto}
I codici di confronto sono molto importanti, poiché in accoppiata con i salti condizionati permettono di costruire tutti i costrutti fondamentali che possiamo trovare anche nei linguaggi di alto livello
\begin{lstlisting}
    CMP #5,D0       * Confronta D0 con 5 (D0 - 5, senza 
                    * modificare D0, aggiorna i flag)
    CMP.W D0,D1     * Confronta D1 con D0 (D1 - D0, 
                    * aggiorna solo i flag)
    CMP.L (A0),D0   * Confronta D0 con il valore in 
                    * memoria puntato da A0
    CMPI #10,D0     * Confronto immediato con D0
                    * (D0 - 10, aggiorna solo i flag)
    CMPA.L #1000,A0 * Confronta registro indirizzo A0 con 1000
\end{lstlisting}

Oltre a semplici comparazioni, solitamente, vi sono anche dei comandi che operano sui singoli registri. Non solo per il controllo di tali registri ma anche per l'effettuazione di eventuali operazioni che possono essere comode per una tipologia di interpretazione ad alto livello del codice
\begin{lstlisting}
    TST D0          * Testa D0 (controlla se e' zero o negativo, 
                    * senza modificarlo)
    TST.W (A0)      * Testa il valore in memoria puntato da A0
    BTST #3, D0     * Testa il bit 3 di D0 
                    * (imposta Z se il bit e' 0)
    BTST #5, (A0)   * Testa il bit 5 della memoria puntata da A0
    BSET #3, D0     * Imposta il bit 3 di D0 a 1
    BSET #5, (A0)   * Imposta il bit 5 della memoria puntata da A0
    BCLR #3, D0     * Azzera il bit 3 di D0
    BCLR #5, (A0)   * Azzera il bit 5 del registro puntato da A0
    BCHG #3, D0     * Inverte il bit 3 di D0 (0 -> 1, 1 -> 0)
    BCHG #5, (A0)   * Inverte il bit 5 del registro puntato da A0
    TAS D0          * Testa e imposta il bit piu' alto (7) di D0
    TAS (A0)        * Testa e imposta il bit 7 del valore in 
                    * memoria puntato da A0
\end{lstlisting}

\subsection{Strutture sintattiche fondamentali}
Dati i codici di \textbf{Salto} [\ref{par:salto}] e quelli di \textbf{Confronto} [\ref{par:confronto}], si possono costruire quelle che sono le struttura sintattiche fondamentali

\subsubsection{if-then-else}
Per costruire il ciclo if-then-else bisogna per prima cosa comprendere quale sia la condizione, poichè bisognerà identificare:
\begin{itemize}
    \item \textbf{Registro target}: In base a quale registro/operazione devo decretare la condizione?
    \item \textbf{Condizione}: Qual'è la condizione da rispettare?
\end{itemize}

Scelti questi due parametri allora sarò capace di capire quale codice cmp utilizzare ed in che modo, e quale tipologia di salto condizionato effettuare.
Esempio di un classico If-Then:
\begin{lstlisting}
        MOVE.L  D0, D1  * Carica valori nei registri 
                        * (supponiamo che D0 e D1 abbiano   gia' valori)
        CMP.L   D1, D0  * Confronta D0 con D1 (D0 - D1)
        BGT     END_IF  * Se D0 > D1, salta al blocco then 
                        * (condizione = (D1 <= D0))
THEN                    * Codice interno all'IF
END_IF                  * Codice successivo...
\end{lstlisting}

Esempio di un If-Then-Else
\begin{lstlisting}
        MOVE.L  D0, D1  * Carica valori nei registri (supponiamo che D0 
                        * e D1 abbiano   gia' valori)
        CMP.L   D1, D0  * Confronta D0 con D1 (D0 - D1)
        BGT     THEN    * Se D0 > D1, salta al blocco THEN
ELSE    MOVE.L  #0, D2  * D2 = 0
        BRA     END_IF  * Salta oltre il blocco THEN 
THEN    MOVE.L  #1, D2  * D2 = 1
END                     * Codice successivo...
\end{lstlisting}


\subsubsection{Ciclo FOR}
Come per l'if il ciclo for è composto principalmente da codici di \textbf{salto} e da codici di \textbf{confronto}. La struttura è molto simile a quella dell'If-Then-Else con l'eccezzione della posizione dei vari salti.
Precisamente la struttura di un ciclo for è la seguente:
\begin{lstlisting}
    MOVE.L  #0, D0  * Inizializza il contatore D0 = 0

FOR CMP.L   #10, D0 * Confronta D0 con 10
    BGE     END     * Se D0 >= 10, esce dal ciclo

                    * Corpo del ciclo ...
    ADDQ.L  #1, D0  * Incrementa D0 di 1
    BRA     FOR     * Ripete il ciclo
END                 

\end{lstlisting}

\subsubsection{Ciclo While}

Il ciclo while segue le regole del ciclo for solo con una condizione differente:
\begin{lstlisting}
WHILE   CMP.L   #0, D1  * Confronta D1 con 0
        BLE     END     * Se D1 <= 0, esce dal ciclo
                        * Corpo del ciclo
        SUBQ.L  #1, D1  * Decrementa D1 di 1
        BRA     WHILE   * Ripete il ciclo
END                     * Codice successivo
\end{lstlisting}


\subsubsection{Chiamata a subroutine}

Le chiamate a subroutine possono essere viste come una sorta di chiamate a funzione. Esse, quindi, possono avere sia degli operandi di ingresso che degli operandi di uscita. La "comunicazione" degli operandi con la subroutine può avvenire in due principali modi:
\begin{itemize}
    \item \textbf{Con registri interni}: Gli operandi vengono caricati nei registri interni prima di chiamare la subroutine, che poi ci lavorerà sopra. Quindi i registri interni vengono utilizzati come una sorta di canale di comunicazione.
    \item \textbf{Con Stack}: Gli operandi sono allocati sullo stack, ciò richiede quindi una gestione anche del puntatore dello stack SP.
\end{itemize}

Un esempio di chiamata a subroutine con memorizzazione degli operandi nello stack è il seguente:
\begin{lstlisting}
        MOVE.L  #5, D0      * Carica il primo operando in D0
        MOVE.L  #10, D1     * Carica il secondo operando in D1
        MOVE.L  D0, -(A7)   * Push del primo operando nello stack
        MOVE.L  D1, -(A7)   * Push del secondo operando
        JSR     SUM_SUB     * Chiamata alla subroutine
        MOVE.L  (A7)+, D2   * prelievlo del risultato dallo stack
        ADDQ.L  #8, A7      * Pulizia dello stack 
                            * (2 valori da 4 byte)
                            * D2 ora contiene il risultato
                            * Codice successivo eventuale
SUM_SUB MOVE.L  (A7)+, D0   * Pop del primo operando dallo stack
        MOVE.L  (A7)+, D1   * Pop del secondo operando
        ADD.L   D1, D0      * Somma D0 + D1, risultato in D0
        MOVE.L  D0, -(A7)   * Push del risultato nello stack
        RTS                 * Ritorna al chiamante
\end{lstlisting}

Esempio di utilizzo dei registri interni, sia per passaggio operandi di ingresso che di uscita:

\begin{lstlisting}
        MOVE.L  #5, D0       * Primo operando in D0
        MOVE.L  #10, D1      * Secondo operando in D1

        JSR     SUM_REGS     * Chiamata alla subroutine
        * Dopo il ritorno, il risultato e' in D0
        * Codice successivo...
SUM_REGS:   ADD.L   D1, D0   * Somma D0 + D1, risultato in D0
        RTS                  * Ritorna al chiamante

\end{lstlisting}

\subsection{Valutazione degli accessi in memoria}
Utilizzando i comandi precedentemente presentati, il processore può accedere in memoria principale. Gli accessi in memoria dipendono fortemente dalla tipologia di architettura che ho adottato. In generale gli accessi in memoria possono avvenire per 2 tipologie di operazioni: Accesso in memoria per le Istruzioni (PI) o accesso in memoria Per le Operazioni (PO).
Veodiamo degli esempi per capire meglio di cosa si sta parlando:
\begin{table}[h]
    \centering
    \begin{tabular}{|c|c|c|}
        \hline
        \textbf{Istruzione} & \textbf{PI} & \textbf{PO} \\
        \hline
        \lstinline|MOVE.L D0,D1| & 1 & 0 \\
        \lstinline|MOVE.W D0,D1| & 1 & 0 \\
        \lstinline|MOVE.L #7,D1| & 3 & 0 \\
        \lstinline|MOVE.W (A0),(A1)| & 1 & 2 \\
        \lstinline|MOVE.W (A0),VAR| & 3 & 2 \\
        \hline
    \end{tabular}
    \caption{Conteggio accessi per Architettura a 16 bit e VAR a 32 bit}
    \label{tab:esempio}
\end{table}

Per contare gli accessi in memoria bisogna effettuare delle osservazioni in base alla parte che si sta analizzando.
Per l'accesso \textbf{Per le Istruzioni (PI)} si ragiona sui seguenti accessi:
\begin{itemize}
    \item \textbf{Prelievo dell'istruzione}: Un operazione che non mancherà mai sarà sempre il prelievo dell'istruzione, che impone che il mio PI non potrà mai essere nullo;
    \item \textbf{Operandi immediati}: se nel mio comando ho degli operandi immediati, allora dovrò accedere anche altre volte alla memoria per il prelievo di tale operando. I miei accessi, per questo caso, sono dettati dalla lunghezza dell'operando rispetto ai miei bus disponibili. Per esempio, se sono in un architettura a 16 bit devo prelevare un immegiato considerato una WORD, allora il numero di accessi aggiuntivi per prelevare l'immediato è uguale a 1. Mentre se stessi lavorango con le Long World (32 bit), allora il numero di accessi, a parità di architettura, sarà 2;
    \item \textbf{Variabili}: se sto utilizzando delle variabili, che indicano delle locazioni di memoria dirette, allora dovrò fare un numero di accessi alla memoria che mi permette di prelevare gli indirizzi (tali indirizzi sono lunghi tutti 32 bit). Pertanto con un architettura a 16 bit dovrò considerare sempre 2 accessi per ogni variabile per prelevare tali indirizzi;
\end{itemize}

Per l'accesso \textbf{Per gli Operandi (PO)}, i parametri risultano più o meno gli stessi, ad esclusione del prelievo istruzione e della variabile. Le operazioni che si contano per tale processo sono:
\begin{itemize}
    \item \textbf{Indirizzamento Indiretto}: quando vado ad effettuare dei riferimenti a dei registri della memoria con indirizzamento indiretto allora devo prevedere un numero di accessi per il prelievo dell'operando. Quindi bisogna conteggiare il numero di accessi per il prelievo dell'operando in base alla sua lunghezza. Nel caso di architettura a 16 bit si avranno: 1 accesso per prelevare delle word (16 bit) e 2 accessi pre prelevare le Long Word (32 bit)

    \item \textbf{Variabili}: Quando si utilizzano le variabili, oltre a prelevare gli indirizzi dalla "zona istruzioni" bisogna prelevare gli operando. La conta del numero di accessi per il prelievo degli operandi è uguale al caso di indirizzamento indiretto
\end{itemize}

Nel caso degli accessi PO, si è prevista la considerazione per singoli operandi. Quindi in base al numero di operandi presenti, la loro lunghezza prefissata, il loro modo di indirizzamento, si riesce a comprendere (cumulando le specifiche), il numero di accessi che bisogna effettuare in memoria ed il motivo di tali accessi.
