\newpage
\section{USART (Universal Synchronous-Asyncronous Receiver-Transmitter)}

L'USART (Universal Synchronous-Asyncronous Receiver-Transmitter) è un interfaccia di comunicazione seriale. In generale i sistemi di comunicazione seriale utilizzano solo 1 filo per comunicare (Tx -> Rx), che permette il collegamento tra il trasmittente ed il ricevente, a questo singolo collegamento, solitamente, si aggiungono altri fili, che aggiungono: controlli di flusso, segnali di tempificazione ecc.
Non si scende troppo nei dettagli di tali architetture, poichè poi si vanno a specializzare nei vari dispositivi che vengono prodotti, un esempio sarà l'Intel 8251.

\subsection{Comunicazioni Sincrona ed Asincrona}
Come detto nel precedente paragrafo, l'USART (a differenza del suo predecessore, l'UART), prevede 2 modalità di funzionamento:
\begin{itemize}
    \item \textbf{Sincrona}
    \item \textbf{Asincrona}
\end{itemize}
Tali modalità sono profondamente diverse, sia per quanto riguarda l'architettura sia per quanto riguarda il loro modo di comunicazione dei dati.

\subsubsection{Comunicazione Sincrona}
Nel caso della comunicazione di tipo Sincrona, i due dispositivi (per una comunicazione unidirezionale) condividono 2 collegamenti:
\begin{itemize}
    \item Tx -> Rx
    \item clk\_t -> clk\_r
\end{itemize}

Questo perchè una comunicazione sincrona sfrutta un segnale di tempificazione comune. La velocità di trasmissione dei dati è quindi dettata dalla frequenza del clock comune. A differenza della comunicazione \textbf{Asincrona}, la comunicazione sincrona è molto più veloce e robusta (data la sincronizzazione in hardware tramite il clock condiviso), ma è molto difficile da implementare a livello hardware, poichè bisogna sincronizzare in maniera molto precisa dati e trasmissione di essi.

\subsubsection{Comunicazione Asincrona}
Nel caso della comunicazione di tipo Asincrona, i due dispositivi richiedono un singolo collegamento (sempre ragionando sul singolo canale trasmissivo Tx -> Rx).
L'invio dei dati è gestito da due bit di start e stop che vanno a scandire, precisamente, l'inizio dell'invio del messaggio e la fine del messaggio.
La tipologia di comunicazione, rispetto alla \textbf{Sincrona} risulta meno efficiente rispetto al trasporto di informazioni, anche se conserva il vantaggio per il rispetto dell'asincronicità intrinseca dei dati, e quindi ottimizza meglio i tempi di invio dei dati (non deve aspettare un fronte del clock condiviso come nel caso della comunicazione sincrona)

\subsubsection{Errori principali presenti nella comiunicazione seriale}
Quando si fanno interagire due sistemi mediante una comunicazione di tipo seriale asincrona (UART), in fase di ricezione, si può andare in contro ai seguenti tipi di errori:
\begin{itemize}
    \item \textbf{Errore di parità}: Il valore del bit di parità inviato non corrisponde con quello calcolato al ricevitore
    \item \textbf{Errore di framing}: Non è previsto il bit di stop
    \item \textbf{Errore di Overruning}: Errore dovuto alla differenza dei clock di invio. Se il sistema ricevente è più lento del sistema trasmittente, allora si può avere perdita di informazioni (non si coglie bene il bit di start)
\end{itemize}

Tra questi errori l'unico su cui si può attuare un miglioramento è quello di overrun, questo perchè si può prevedere che il dato venga trasmesso secondo uno specifico protocollo di handshacking. Difatti nelle architetture sono presenti appositi collegamenti per l'implementazione di tale soluzione, anche se quest'ultima rallenta un po la ricezione rendendola meno efficente.
